\documentclass[a4paper]{article}

%% Language and font encodings
\usepackage[french]{babel}
\usepackage[utf8]{inputenc}
\usepackage[T1]{fontenc}

\usepackage{float}

\setlength{\parindent}{1em}
%\setlength{\parskip}{1ex plus 0.5ex minus 0.2ex}
\newcommand{\hsp}{\hspace{20pt}}
\newcommand{\HRule}{\rule{\linewidth}{0.5mm}}

\usepackage{algorithm}
\usepackage[noend]{algpseudocode}
\algnewcommand{\algorithmicand}{\textbf{ and }}
\algnewcommand{\algorithmicor}{\textbf{ or }}
\algnewcommand{\OR}{\algorithmicor}
\algnewcommand{\AND}{\algorithmicand}
\algnewcommand\algorithmicforeach{\textbf{for each}}
\algdef{S}[FOR]{ForEach}[1]{\algorithmicforeach\ #1\ \algorithmicdo}
\newcommand{\myfrac}[2]{\frac{\displaystyle {#1}}{\displaystyle {#2}}}

%% Sets page size and margins
\usepackage[a4paper,top=3cm,bottom=2cm,left=3cm,right=3cm,marginparwidth=1.75cm]{geometry}

%% Useful packages
\usepackage{amsmath}
\usepackage{amssymb}
\usepackage{graphicx}
\usepackage{subcaption}
\usepackage[colorinlistoftodos]{todonotes}
\usepackage[colorlinks=true, allcolors=blue]{hyperref}
\usepackage{graphicx}

\usepackage{enumitem}
\setitemize{label=\textbullet, font=\small}

%% equations
\usepackage{amsthm}
\usepackage[retainorgcmds]{IEEEtrantools}

%% languages
\usepackage{listingsutf8}
\lstset{language=Python}
\lstset{inputencoding=utf8/latin1}

%% question numeration
\newcommand{\question}[1]{\noindent \textbf{#1}~}

%% theorem and proposition
\newtheorem{prop}{Proposition}
\newtheorem*{prop*}{Proposition}
\newtheorem{thm}{Théorème}

\newenvironment{myproof}[1][\proofname]{\proof[#1]\mbox{}\\*}{\endproof}

%% references shortcuts (Arthur) 
\usepackage{suffix}
\renewcommand{\eqref}[1]{équation~\ref{#1}}
\newcommand{\algoref}[1]{algorithme~\ref{#1}}
\newcommand{\figref}[1]{figure~\ref{#1}}
\newcommand{\tabref}[1]{tableau~\ref{#1}}
\newcommand{\secref}[1]{section~\ref{#1}}
\newcommand{\probref}[1]{problème~\ref{#1}}
\newcommand{\propref}[1]{proposition~\ref{#1}}
\newcommand{\theoremref}[1]{théorème~\ref{#1}}
\newcommand{\chapref}[1]{chapitre~\ref{#1}}
\WithSuffix\newcommand\algoref*[1]{algorithme~\ref{#1} p.~\pageref{#1}}
\WithSuffix\newcommand\figref*[1]{figure~\ref{#1} p.~\pageref{#1}}
\WithSuffix\newcommand\eqref*[1]{équation~\ref{#1} p.~\pageref{#1}}
\WithSuffix\newcommand\tabref*[1]{tableau~\ref{#1} p.~\pageref{#1}}
\WithSuffix\newcommand\secref*[1]{section~\ref{#1} p.~\pageref{#1}}
\WithSuffix\newcommand\probref*[1]{problème~\ref{#1} p.~\pageref{#1}}
\WithSuffix\newcommand\propref*[1]{proposition~\ref{#1} p.~\pageref{#1}}
\WithSuffix\newcommand\chapref*[1]{chapitre~\ref{#1} p.~\pageref{#1}}

%% equations
\usepackage{amsthm}
\usepackage[retainorgcmds]{IEEEtrantools}

\usepackage[backend=biber,uniquename=init,giveninits=true,
             %% "et al" pour > deux auteurs, & pour exactement 2
             uniquelist=false,maxcitenames=2,mincitenames=1,maxbibnames=99,
             isbn=false,url=false,doi=false,bibstyle=numeric
]{biblatex}
\addbibresource{references.bib}

\begin{document}

\begin{titlepage}
  \begin{center}

      \makebox[0.5\textwidth][r]{%
        \includegraphics[width=0.33\textwidth]{images/sorbonne.png}%
    }%

      \vspace{4cm}
    % Title
    \HRule \\[0.4cm]
    { \huge \bfseries COMPLEX - PROJET\\[0.4cm] }

      \textsc{\LARGE Tests de primalité}\\[0.4cm]

    \HRule \\[0.8cm]

    % Author and supervisor
    \begin{minipage}{0.4\textwidth}
      \begin{flushleft} \large
        Fatemeh \textsc{Hamissi}\\
        \large
        Kim-Anh Laura \textsc{Nguyen}\\
        Promo 2018-2019 \\
      \end{flushleft}
    \end{minipage}
    \begin{minipage}{0.5\textwidth}
      \begin{flushright} \large
        \emph{Chargé de TP :} Alexandre \textsc{Teiller}\\
      \end{flushright}
    \end{minipage}

      \vspace{2cm}

  \end{center}
  %\end{sffamily}
\end{titlepage}
%\maketitle

\section*{2. Test naïf et recherche des nombres de Carmichael}

\question{a)}
\begin{lstlisting}[frame=single]  % Start your code-block

def first_test(n):
    '''
        n : entier
        effectue le test naïf de primalité sur n
    '''
    if n<7:
        if n in (2,3,5):
            return True
        else:
            return False
    if n & 1 == 0:
        return False
    k=3
    r=math.sqrt(n)
    while k<=r:
        if n % k == 0:
            return False
        k+=2
    return True
\end{lstlisting}

\question{b)} \texttt{first\_test} est exponentiel en la taille de l'entrée.\\

\question{c)} \\
\question{d)} \\
\question{f)} \\

\question{g)} Montrons qu'il n'existe qu'un nombre fini de nombres de Carmichael
de la forme $pqr$ avec $p < q < r$ trois nombres premiers pour $p$ fixé. \\

1. Soit $n$ un nombre de Carmichael de la forme $pqr$ avec $p < q < r$ trois
nombres premiers. Montrons qu'il existe un entier $h \in \{2,...,p-1\}$ tel que
$(pq - 1) = h(r - 1)$.

\begin{myproof}
D'après le critère de Korselt, on a :

$$ p-1\ |\ pqr-1$$
$$ q-1\ |\ pqr-1$$
$$ r-1\ |\ pqr-1$$

d'où 

\begin{equation} p-1\ |\ qr-1\label{div_p}\end{equation}
\begin{equation} q-1\ |\ pr-1\label{div_q}\end{equation}
\begin{equation} r-1\ |\ pq-1\label{div_r}\end{equation}

donc il existe un entier $h$ tel que $(pq-1) = h(r-1)$. De plus, $q$ et $r$ ne
peuvent pas être consécutifs car premiers, d'où $q < r-1$. On a alors $qh < pq$
et donc $h < p$. Comme $h$ et $p$ sont des entiers, on obtient $h \leq p-1$. On a
aussi $h > 1$ car $pq - 1 \neq r - 1$ car $pq \neq r$ car $r$ est premier. Donc
$h \in \{2,...,p-1\}$.\\

\end{myproof}

2. Montrons qu'il existe un entier $k$ tel que $(hk-p^2)(q-1) = (p+h)(p-1)$. 

\begin{myproof}

D'après l'\eqref{div_q}, il existe $k$ tel que $pr-1 = k(q-1)$. On a donc
\begin{IEEEeqnarray*}{rCl"s}
    k(q-1) &=& pr-1 \\
    &=& p(r-1) + (p-1) \\
    &=& p\ \frac{pq-1}{h} + (p-1) \\ 
    &=& \frac{p}{h} \ (pq-1) + (p-1)\\
\end{IEEEeqnarray*}

d'où 
\begin{IEEEeqnarray*}{rCl"s}
    hk(q-1) &=& p (pq-1) + h(p-1)\\
    hkq-hk&=& p^2q - p + hp - h\\
\end{IEEEeqnarray*}

et 
\begin{IEEEeqnarray*}{rCl"s}
    (hk-p^2)(q-1) &=& (p+h)(p-1)\\ 
    hkq - hk - p^2q + p^2 &=& p^2-p+hp-h\\
    hkq -hk - p^2q &=& -p + hp - h\\
    hkq-hk&=& p^2q - p + hp - h\\
\end{IEEEeqnarray*}

    donc il existe un entier $k$ tel que $(hk-p^2)(q-1) = (p+h)(p-1)$.

\end{myproof}

3. Montrons que tout nombre de Carmichael $n = pqr$.

\begin{myproof}

    On sait que $q = \frac{(p+h)(p-1)}{(hk-p^2)} + 1$. Alors,

    $$q \leq (p+h)(p-1) + 1$$



\end{myproof}
\end{document}
